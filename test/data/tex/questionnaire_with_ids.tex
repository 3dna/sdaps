\documentclass[pdf, print_questionnaire_id, globalid=SDAPS, ngerman, stamp, pagemark]{sdaps}
\usepackage{ifxetex}
\ifxetex
\else
  \usepackage[utf8]{inputenc}
\fi
\usepackage{ulem}
\usepackage{babel}

\author{Someone}
\title{Testfragebogen \LaTeX}

\begin{document}
  % Will be printed on the report
  \begin{questionnaire}
    \addinfo{Umfrage}{Testfragebogen}

    \begin{info}
      Hier steht ein Informationstext, denn man ganz normal, \textbf{Fett}, \textit{kursiv}, oder \underline{unterstreichen} setzen kann.\par
      \centering Auch das Zentrieren des Textes ist möglich.
    \end{info}

    \section{Bewertungsfragen}
    \singlemark{Einzeln stehende Frage}{sehr gut}{sehr schlecht}

    \begin{markgroup}{Mehrere gebündelte Bewertungsfragen.}
      \markline{stolz}{sollte man haben}{finde ich unangenehm}
      \markline{hier ist ein langer text. Lorem ipsum dolor sit amet,
      consectetur adipiscing elit.}{finde ich gut}{finde ich voll doof}
      \markline{nett}{wichtig}{unwichtig}
      \markline{sympatisch}{wichtig}{unwichtig}
    \end{markgroup}

    \section{Fragen mit Auswahlfeldern}
    \begin{choicequestion}[4]{Bitte wähle etwas aus oder schreibe in das Textfeld}
      \choiceitem{erste wahl}
      \choiceitem{zweite wahl}
      \choiceitem{dritte wahl}
      \choiceitem{vierte wahl}
      \choiceitem{fünfte wahl}
      \choiceitem{sechste wahl}
      \choicemulticolitem{2}{längere Auswahl mit langem Text damit mans merkt}
      \choiceitem{siebte wahl}
      \choiceitemtext{1cm}{2}{Sonstiges:}
    \end{choicequestion}

    Alternativ eine Liste von Auswahlfragen mit den gleichen Antworten.
    \begin{choicegroup}{Welche Software ist für die folgenden Anwendungen am besten geeignet?}
      \groupaddchoice{\LaTeX}
      \groupaddchoice{LibreOffice}
      \groupaddchoice{Microsoft Word}

      \choiceline{Texte schreiben}
      \choiceline{Mathematische Formeln}
      \choiceline{Fragebögen erstellen}
    \end{choicegroup}

    \section{Freitextfelder}
    Freitextfelder werden automatisch in der höhe Skaliert so dass die Seite voll wird.
    Es muss aber eine Mindesthöhe angegeben werden.
    \textbox{2cm}{Hier sollst du was eintragen}
    \textbox{4cm}{Und mal etwas mit macros \LaTeX}

    Und noch eine andere Frage.
    \singlemark{Wie gefällt dir dieser Bogen?}{sehr gut}{sehr schlecht}
  \end{questionnaire}
\end{document}
